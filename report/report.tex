\documentclass{article}

% if you need to pass options to natbib, use, e.g.:
%     \PassOptionsToPackage{numbers, compress}{natbib}
% before loading neurips_2021

% ready for submission
\usepackage[preprint]{neurips_2021}

% to compile a preprint version, e.g., for submission to arXiv, add add the
% [preprint] option:
%     \usepackage[preprint]{neurips_2021}

% to compile a camera-ready version, add the [final] option, e.g.:
%     \usepackage[final]{neurips_2021}

% to avoid loading the natbib package, add option nonatbib:
%    \usepackage[nonatbib]{neurips_2021}

\usepackage[utf8]{inputenc} % allow utf-8 input
\usepackage[T1]{fontenc}    % use 8-bit T1 fonts
\usepackage[colorlinks=true]{hyperref}       % hyperlinks
\usepackage{url}            % simple URL typesetting
\usepackage{booktabs}       % professional-quality tables
\usepackage{amsfonts}       % blackboard math symbols
\usepackage{nicefrac}       % compact symbols for 1/2, etc.
\usepackage{microtype}      % microtypography
\usepackage{xcolor}         % colors

\title{Historical topic shifts in German parliament speeches}

% The \author macro works with any number of authors. There are two commands
% used to separate the names and addresses of multiple authors: \And and \AND.
%
% Using \And between authors leaves it to LaTeX to determine where to break the
% lines. Using \AND forces a line break at that point. So, if LaTeX puts 3 of 4
% authors names on the first line, and the last on the second line, try using
% \AND instead of \And before the third author name.

\author{%
  Emanuel Fuchs\\
  Matrikelnummer xxxxxxx\\
  \texttt{emanuel-fuchs@t-online.de} \\
  \And
  Arthur Jaques\\
  Matrikelnummer xxxxxxx\\
  \texttt{arthur.jaques@live.com} \\
}

\begin{document}

\maketitle

\begin{abstract}
  We are planning to use the dataset of speech transcripts of the German Bundestag (\url{https://de.openparliament.tv/api/}).
  Our main research question is how the topics of the speeches have changed over time, both in general and by party.
  We propose to use Latent Dirichlet Allocation (LDA) for topic extraction and visualize the results both with simple barplots and time series visualizations.
\end{abstract}

\section{Introduction}

\section{Methods}
Data collection, preprocessing, LDA, PCA, sentiment analysis, p-value testing.

\section{Results}
\begin{itemize}
  \item Topic-words.
  \item Stacked plot.
  \item Topic example(s).
  \item Party PCA. Mention differences.
  \item Sentiment colormap.
  \item Sentiment p-value testing.
\end{itemize}

\section{Discussion and conclusion}

\end{document}
