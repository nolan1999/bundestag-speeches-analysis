\documentclass{article}

% if you need to pass options to natbib, use, e.g.:
%     \PassOptionsToPackage{numbers, compress}{natbib}
% before loading neurips_2021

% ready for submission
\usepackage[preprint]{neurips_2021}

% to compile a preprint version, e.g., for submission to arXiv, add add the
% [preprint] option:
%     \usepackage[preprint]{neurips_2021}

% to compile a camera-ready version, add the [final] option, e.g.:
%     \usepackage[final]{neurips_2021}

% to avoid loading the natbib package, add option nonatbib:
%    \usepackage[nonatbib]{neurips_2021}

\usepackage[utf8]{inputenc} % allow utf-8 input
\usepackage[T1]{fontenc}    % use 8-bit T1 fonts
\usepackage[colorlinks=true]{hyperref}       % hyperlinks
\usepackage{url}            % simple URL typesetting
\usepackage{booktabs}       % professional-quality tables
\usepackage{amsfonts}       % blackboard math symbols
\usepackage{nicefrac}       % compact symbols for 1/2, etc.
\usepackage{microtype}      % microtypography
\usepackage{xcolor}         % colors

\title{Semantic analysis of German parliament speeches}

% The \author macro works with any number of authors. There are two commands
% used to separate the names and addresses of multiple authors: \And and \AND.
%
% Using \And between authors leaves it to LaTeX to determine where to break the
% lines. Using \AND forces a line break at that point. So, if LaTeX puts 3 of 4
% authors names on the first line, and the last on the second line, try using
% \AND instead of \And before the third author name.

\author{%
  Emanuel Fuchs\\
  Matrikelnummer xxxxxxx\\
  \texttt{emanuel-fuchs@t-online.de} \\
  \And
  Arthur Jaques\\
  Matrikelnummer xxxxxxx\\
  \texttt{arthur.jaques@live.com} \\
}

\begin{document}

\maketitle

\begin{abstract}
  We are planning to use the dataset of speech transcripts of the German Bundestag (\url{https://de.openparliament.tv/api/}).
  Our main research question is how the topics of the speeches have changed over time, both in general and by party.
  We propose to use Latent Dirichlet Allocation (LDA) for topic extraction and visualize the results both with simple barplots and time series visualizations.
\end{abstract}

\section{Introduction}

\section{Methods}
We download the texts of the Bundestag speeches in the timeframe 24 October 2017 to 7 September 2021 using the OpenParliament \cite{OpenParliamentTV} API.
We discard all speeches not attributed to any of the major parliament factions (SPD, FDP, CDU/CSU, BÜNDNIS 90/DIE GRÜNEN, AfD, DIE LINKE).
We remove as many ``formalities'' (expressions empty of content such as salutations) as possible, using regular expressions.

For topic extraction, we use Scikit-learn's \cite{Scikit-learn} implementation of Latent Dirichlet Allocation with default parameters,  trained on the whole dataset and used to extract topics from each speech.
The features are built by stemming the texts using Nltk's \cite{Nltk} Snowball German Stemmer and vectorizing using Scikit-learn's CountVectorizer \cite{Scikit-learn}.
Tf-idf features were tried but led to less interpretable results.
After topic extraction, we analyze the words having the highest weights for each topic and try to name the topics accordingly.
We tried different parameters for vectorization and topic extraction methods (Non-negative Matrix Factorization) and decided on the parameters that yielded the best interpretable results (a subjective criterion).
In the end, we opted for 9 topics and keeping only words appearing in at least 20 speeches and at most 30\% of the total amount of speeches for vectorization.
For the visualization of time series, we use Gaussian smoothing.
To plot the average party topic distributions, we reduce the dimensionality to 2 dimensions using Principal Component Analysis.

Finally, we extract sentiments with the method provided the Python package germansentiment \cite{Germansentiment}, which uses the Bert architecture trained on German texts.
Since only a very small subset of the speeches are assigned a positive sentiment (most being neutral), we concentrate our analysis on the fraction of negative speeches.
We use p-value testing with a binomial assumption for the distribution of negative sentiments to test for significant asymmetrical differences between parties in the proportion of negative speeches.

\section{Results}
The topics extracted from the speeches dataset and the name that was assigned to them are summarized in Table~\ref{topics_table}.
\begin{table}
  \caption{Extracted topics}
  \label{topics_table}
  \centering
  \begin{tabular}{p{0.02\linewidth} | p{0.2\linewidth} | p{0.78\linewidth}}
    \toprule
    \# & Assigned name & Strongest predictors \\
    \midrule
    1 & International & europa; deutsch; russland; staat; gemeinsam; international; eu; menschenrecht; welt; turkei. \\
    2 & Military & soldat; einsatz; bundeswehr; mandat; soldatinn; mission; afghanistan; unterstutz; mali; militar. \\
    3 & EU/Economy & europa; euro; eu; unternehm; milliard; deutsch; prozent; union; geld; wirtschaft. \\
    4 & Social & kind; euro; famili; prozent; arbeit; sozial; 000; rent; miet; hoh. \\
    5 & Decisions/Law & gesetzentwurf; bundestag; glaub; fall; thema; hatt; word; entscheid; punkt; regel. \\
    6 & Democracy/Freedom & emokrati; leb; gesellschaft; freiheit; grundgesetz; staat; gewalt; deutsch; demokrat; wer. \\
    7 & German History & deutsch; wer; ost; geschicht; stiftung; abstimm; 19; opf; stimmt; bitt. \\
    8 & Ecology & klimaschutz; co; energi; prozent; erneuerbar; bau; ziel; verbrauch; energiew; wirtschaft. \\
    9 & Health/Pandemic & pandemi; schul; unternehm; arbeit; stark; digital; bass; bereich; massnahm; wirtschaft. \\
    \bottomrule
  \end{tabular}
\end{table}


\begin{itemize}
  \item Topic-words.
  \item Stacked plot.
  \item Topic example(s).
  \item Party PCA. Mention differences.
  \item Sentiment colormap.
  \item Sentiment p-value testing.
\end{itemize}

\section{Discussion and conclusion}
Limitations:
\begin{itemize}
  \item Data: incomplete data, aggressive procedure to get data, reliance of data
  \item Visualization: smoothing (free parameters), short time frame
  \item Topic extraction: free parameters, subjectivity of chosen topic names
  \item Sentiment analysis: black-box model
\end{itemize}

\bibliographystyle{plain}
\bibliography{bibliography}

\end{document}
